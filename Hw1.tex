%% HOW TO USE THIS TEMPLATE:
%%
%% Ensure that you replace "YOUR NAME HERE" with your own name, in the
%% \studentname command below.  Also ensure that the "answers" option
%% appears within the square brackets of the \documentclass command,
%% otherwise latex will suppress your solutions when compiling.
%% 
%% Type your solution to each problem part within
%% the \begin{solution} \end{solution} environment immediately
%% following it.  Use any of the macros or notation from the
%% header.tex that you need, or use your own (but try to stay
%% consistent with the notation used in the problem).
%%
%% If you have problems compiling this file, you may lack the
%% header.tex file (available on the course web page), or your system
%% may lack some LaTeX packages.  The "exam" package (required) is
%% available at:
%%
%% http://mirror.ctan.org/macros/latex/contrib/exam/exam.cls
%%
%% Other packages can be found at ctan.org, or you may just comment
%% them out (only the exam and ams* packages are absolutely required).


% the "answers" option causes the solutions to be printed
%\documentclass[11pt,addpoints]{exam}
\documentclass[11pt,addpoints,answers]{exam}


% required macros -- get header.tex file from course web page

\usepackage{amsmath,amsfonts,amssymb,amsthm}
\usepackage{fullpage}
\usepackage{times}
\usepackage{hyperref}
\usepackage{relsize}
%\usepackage{pdfsync}
%\usepackage{microtype}
\usepackage{color}
%\usepackage{cleveref}
%\crefformat{footnote}{#2\footnotemark[#1]#3}
%\definecolor{light-gray}{gray}{0.5}

%%% BLACKBOARD SYMBOLS

\newcommand{\C}{\ensuremath{\mathbb{C}}}
\newcommand{\D}{\ensuremath{\mathbb{D}}}
\newcommand{\F}{\ensuremath{\mathbb{F}}}
\newcommand{\G}{\ensuremath{\mathbb{G}}}
\newcommand{\J}{\ensuremath{\mathbb{J}}}
\newcommand{\N}{\ensuremath{\mathbb{N}}}
\newcommand{\Q}{\ensuremath{\mathbb{Q}}}
\newcommand{\R}{\ensuremath{\mathbb{R}}}
\newcommand{\T}{\ensuremath{\mathbb{T}}}
\newcommand{\Z}{\ensuremath{\mathbb{Z}}}
\newcommand{\QR}{\ensuremath{\mathbb{QR}}}

\newcommand{\Zt}{\ensuremath{\Z_t}}
\newcommand{\Zp}{\ensuremath{\Z_p}}
\newcommand{\Zq}{\ensuremath{\Z_q}}
\newcommand{\ZN}{\ensuremath{\Z_N}}
\newcommand{\Zps}{\ensuremath{\Z_p^*}}
\newcommand{\ZNs}{\ensuremath{\Z_N^*}}
\newcommand{\JN}{\ensuremath{\J_N}}
\newcommand{\QRN}{\ensuremath{\QR_{N}}}
\newcommand{\QRp}{\ensuremath{\QR_{p}}}

%%% THEOREM COMMANDS

\theoremstyle{plain}            % following are "theorem" style
\newtheorem{theorem}{Theorem}[section]
\newtheorem{lemma}[theorem]{Lemma}
\newtheorem{corollary}[theorem]{Corollary}
\newtheorem{proposition}[theorem]{Proposition}
\newtheorem{claim}[theorem]{Claim}
\newtheorem{fact}[theorem]{Fact}

\theoremstyle{definition}       % following are def style
\newtheorem{definition}[theorem]{Definition}
\newtheorem{conjecture}[theorem]{Conjecture}
\newtheorem{example}[theorem]{Example}
\newtheorem{protocol}[theorem]{Protocol}

\theoremstyle{remark}           % following are remark style
\newtheorem{remark}[theorem]{Remark}
\newtheorem*{note}{Note}
\newtheorem{exercise}[theorem]{Exercise}

% equation numbering style
\numberwithin{equation}{section}

%%% GENERAL COMPUTING

\newcommand{\bit}{\ensuremath{\set{0,1}}}
\newcommand{\pmone}{\ensuremath{\set{-1,1}}}

% asymptotics
\DeclareMathOperator{\poly}{poly}
\DeclareMathOperator{\polylog}{polylog}
\DeclareMathOperator{\negl}{negl}
\newcommand{\Otil}{\ensuremath{\tilde{O}}}

% probability/distribution stuff
\DeclareMathOperator*{\E}{E}
\DeclareMathOperator*{\Var}{Var}

% sets in calligraphic type
\newcommand{\calD}{\ensuremath{\mathcal{D}}}
\newcommand{\calF}{\ensuremath{\mathcal{F}}}
\newcommand{\calH}{\ensuremath{\mathcal{H}}}
\newcommand{\calK}{\ensuremath{\mathcal{K}}}
\newcommand{\calM}{\ensuremath{\mathcal{M}}}
\newcommand{\calX}{\ensuremath{\mathcal{X}}}
\newcommand{\calY}{\ensuremath{\mathcal{Y}}}

% types of indistinguishability
\newcommand{\compind}{\ensuremath{\stackrel{c}{\approx}}}
\newcommand{\statind}{\ensuremath{\stackrel{s}{\approx}}}
\newcommand{\perfind}{\ensuremath{\equiv}}

% font for general-purpose algorithms
\newcommand{\algo}[1]{\ensuremath{\mathsf{#1}}}
% font for general-purpose computational problems
\newcommand{\problem}[1]{\ensuremath{\mathsf{#1}}}
% font for complexity classes
\newcommand{\class}[1]{\ensuremath{\mathsf{#1}}}

% complexity classes and languages
\renewcommand{\P}{\class{P}}
\newcommand{\BPP}{\class{BPP}}
\newcommand{\NP}{\class{NP}}
\newcommand{\coNP}{\class{coNP}}
\newcommand{\AM}{\class{AM}}
\newcommand{\coAM}{\class{coAM}}
\newcommand{\IP}{\class{IP}}

%%% "LEFT-RIGHT" PAIRS OF SYMBOLS

% inner product
\newcommand{\inner}[1]{\langle{#1}\rangle}
\newcommand{\innerfit}[1]{\left\langle{#1}\right\rangle}
% absolute value
\newcommand{\abs}[1]{\lvert{#1}\rvert}
\newcommand{\absfit}[1]{\left\lvert{#1}\right\rvert}
% a set
\newcommand{\set}[1]{\{{#1}\}}
\newcommand{\setfit}[1]{\left\{{#1}\right\}}
% parens
\newcommand{\parens}[1]{({#1})}
\newcommand{\parensfit}[1]{\left({#1}\right)}
% tuple = alias for parens
\newcommand{\tuple}[1]{\parens{#1}}
\newcommand{\tuplefit}[1]{\parensfit{#1}}
% square brackets
\newcommand{\bracks}[1]{[{#1}]}
\newcommand{\bracksfit}[1]{\left[{#1}\right]}
% rounding off
\newcommand{\round}[1]{\lfloor{#1}\rceil}
% floor function
\newcommand{\floor}[1]{\lfloor{#1}\rfloor}
% ceiling function
\newcommand{\ceil}[1]{\lceil{#1}\rceil}
% length of a string
\newcommand{\len}[1]{\lvert{#1}\rvert}
\newcommand{\lenfit}[1]{\left\lvert{#1}\right\rvert}
% length of some vector, element
\newcommand{\length}[1]{\lVert{#1}\rVert}
\newcommand{\lengthfit}[1]{\left\lVert{#1}\right\rVert}

%%% CRYPTO-RELATED NOTATION

% KEYS AND RELATED

\newcommand{\key}[1]{\ensuremath{#1}}

\newcommand{\pk}{\key{pk}}
\newcommand{\vk}{\key{vk}}
\newcommand{\sk}{\key{sk}}
\newcommand{\mpk}{\key{mpk}}
\newcommand{\msk}{\key{msk}}
\newcommand{\fk}{\key{fk}}
\newcommand{\id}{id}
\newcommand{\keyspace}{\ensuremath{\mathcal{K}}}
\newcommand{\msgspace}{\ensuremath{\mathcal{M}}}
\newcommand{\ctspace}{\ensuremath{\mathcal{C}}}
\newcommand{\tagspace}{\ensuremath{\mathcal{T}}}
\newcommand{\idspace}{\ensuremath{\mathcal{ID}}}

\newcommand{\concat}{\ensuremath{\|}}

% GAMES

% advantage
\newcommand{\advan}{\ensuremath{\mathbf{Adv}}}

% different attack models
\newcommand{\attack}[1]{\ensuremath{\text{#1}}}

\newcommand{\atk}{\attack{atk}} % dummy attack
\newcommand{\indcpa}{\attack{ind-cpa}}
\newcommand{\indcca}{\attack{ind-cca}}
\newcommand{\anocpa}{\attack{ano-cpa}} % anonymous
\newcommand{\anocca}{\attack{ano-cca}}
\newcommand{\euacma}{\attack{eu-acma}} % forgery: adaptive chosen-message
\newcommand{\euscma}{\attack{eu-scma}} % forgery: static chosen-message
\newcommand{\suacma}{\attack{su-acma}} % strongly unforgeable

% ADVERSARIES
\newcommand{\attacker}[1]{\ensuremath{\mathcal{#1}}}

\newcommand{\Adv}{\attacker{A}}
\newcommand{\AdvA}{\attacker{A}}
\newcommand{\AdvB}{\attacker{B}}
\newcommand{\Dist}{\attacker{D}}
\newcommand{\Sim}{\attacker{S}}
\newcommand{\Ora}{\attacker{O}}
\newcommand{\Inv}{\attacker{I}}
\newcommand{\For}{\attacker{F}}

% CRYPTO SCHEMES

\newcommand{\scheme}[1]{\ensuremath{\text{#1}}}

% pseudorandom stuff
\newcommand{\prg}{\algo{PRG}}
\newcommand{\prf}{\algo{PRF}}
\newcommand{\prp}{\algo{PRP}}

% symmetric-key cryptosystem
\newcommand{\skc}{\scheme{SKC}}
\newcommand{\skcgen}{\algo{Gen}}
\newcommand{\skcenc}{\algo{Enc}}
\newcommand{\skcdec}{\algo{Dec}}

% public-key cryptosystem
\newcommand{\pkc}{\scheme{PKC}}
\newcommand{\pkcgen}{\algo{Gen}}
\newcommand{\pkcenc}{\algo{Enc}} % can also use \kemenc and \kemdec
\newcommand{\pkcdec}{\algo{Dec}}

% digital signatures
\newcommand{\sig}{\scheme{SIG}}
\newcommand{\siggen}{\algo{Gen}}
\newcommand{\sigsign}{\algo{Sign}}
\newcommand{\sigver}{\algo{Ver}}

% message authentication code
\newcommand{\mac}{\scheme{MAC}}
\newcommand{\macgen}{\algo{Gen}}
\newcommand{\mactag}{\algo{Tag}}
\newcommand{\macver}{\algo{Ver}}

% key-encapsulation mechanism
\newcommand{\kem}{\scheme{KEM}}
\newcommand{\kemgen}{\algo{Gen}}
\newcommand{\kemenc}{\algo{Encaps}}
\newcommand{\kemdec}{\algo{Decaps}}

% identity-based encryption
\newcommand{\ibe}{\scheme{IBE}}
\newcommand{\ibesetup}{\algo{Setup}}
\newcommand{\ibeext}{\algo{Ext}}
\newcommand{\ibeenc}{\algo{Enc}}
\newcommand{\ibedec}{\algo{Dec}}

% hierarchical IBE (as key encapsulation)
\newcommand{\hibe}{\scheme{HIBE}}
\newcommand{\hibesetup}{\algo{Setup}}
\newcommand{\hibeext}{\algo{Extract}}
\newcommand{\hibeenc}{\algo{Encaps}}
\newcommand{\hibedec}{\algo{Decaps}}

% binary tree encryption (as key encapsulation)
\newcommand{\bte}{\scheme{BTE}}
\newcommand{\btesetup}{\algo{Setup}}
\newcommand{\bteext}{\algo{Extract}}
\newcommand{\bteenc}{\algo{Encaps}}
\newcommand{\btedec}{\algo{Decaps}}

% trapdoor functions
\newcommand{\tdf}{\scheme{TDF}}
\newcommand{\tdfgen}{\algo{Gen}}
\newcommand{\tdfeval}{\algo{Eval}}
\newcommand{\tdfinv}{\algo{Invert}}
\newcommand{\tdfver}{\algo{Ver}}

%%% PROTOCOLS

\newcommand{\out}{\text{out}}
\newcommand{\view}{\text{view}}

%%% COMMANDS FOR LECTURES/HOMEWORKS

\newcommand{\lecheader}{%
  \chead{\large \textbf{Lecture \lecturenum\\\lecturetopic}}

  \lhead{\small
    \textbf{\href{http://www.cims.nyu.edu/~regev/teaching/crypto_fall_2013/}{Introduction to Cryptography}\\Courant, Fall 2013}}

  \rhead{\small \textbf{Instructor:
      \href{http://www.cims.nyu.edu/~regev/}{Oded Regev}\\Scribe:
      \scribename}}

  \setlength{\headheight}{20pt}
  \setlength{\headsep}{16pt}
}

\newcommand{\hwheader}{%
  \chead{\Large \textbf{Tutorial \hwnum}}

  \lhead{\small
    \textbf{\href{}{Engineering Mathematics-II (MA102)}\\GBU, April 2014}}

  %\rhead{\small \textbf{Instructors:{Fahed, Amit, and Amit}\\
  %Student: \studentname
  %}
  %}

  \setlength{\headheight}{20pt}
  \setlength{\headsep}{16pt}
  
  \headrule
}

\newcommand{\asheader}{%
  \chead{\Large \textbf{Assignment \hwnum}}

  \lhead{\small
    \textbf{\href{}{Engineering Mathematics-II (MA102)}\\GBU, April 2014}}

  %\rhead{\small \textbf{Instructors:{Fahed, Amit, and Amit}\\
  %Student: \studentname
  %}
  %}

  \setlength{\headheight}{20pt}
  \setlength{\headsep}{16pt}
  
  \headrule
}


\newcommand{\examheader}{%
  \chead{\Large \textbf{Exam}}
  
  \lhead{\small
    {\textbf{}{Engineering Mathematics}\\GBU, April 2013}}

  \rhead{\small \textbf{Faculty:
     % \href{}{Fahed, Amit, Amit}\\%Student: \studentname
     }}

  \setlength{\headheight}{20pt}
  \setlength{\headsep}{16pt}
  
  \headrule
}
  
        


% VARIABLES

\newcommand{\hwnum}{1}
\newcommand{\duedate}{Sep 16} % changing this does not change the
                                % actual due date :)
\newcommand{\studentname}{YOUR NAME HERE}

% END OF SUPPLIED VARIABLES

\hwheader                       % execute homework commands

\begin{document}

\pagestyle{head}                % put header on every page

\noindent Homework is due by \textbf{noon of 
  \duedate}. Send both TEX and PDF by email to Oded and bring a printed copy to class.  
	Start early!

\medskip
\noindent \textbf{Instructions.} Solutions must be typeset in \LaTeX\
(a template for this homework is available on the course web page).
Your work will be graded on \emph{correctness}, \emph{clarity}, and
\emph{conciseness}.  You should only submit work that you believe to
be correct; if you cannot solve a problem completely, you will get
significantly more partial credit if you clearly identify the gap(s)
in your solution.  It is good practice to start any long solution with
an informal (but accurate) ``proof summary'' that describes the main
idea.

\medskip
\noindent You may collaborate with others on this problem set and
consult external sources.  However, you must \textbf{\emph{write your
    own solutions}} and \textbf{\emph{list your
    collaborators/sources}} for each problem.

% QUESTIONS START HERE.  PROVIDE SOLUTIONS WITHIN THE "solution"
% ENVIRONMENTS FOLLOWING EACH QUESTION.

\begin{questions}
  \question \emph{(Due by noon on Wed, Sep 11.)}  Send email to
  Oded (\texttt{regev} at \texttt{cims}) with subject \texttt{CSCI-GA 3210
    student} containing (1) a few sentences about yourself and your
  background (including your department, graduate program, how long in program), (2)
  what you hope to get out of this course, and (3) your comfort level
  with the following: mathematical proofs, elementary probability
  theory, big-O notation and analysis of algorithms, Turing machines,
  and $\P$, $\BPP$, $\NP$, and $\NP$-completeness.  Please also
  mention any courses you've taken covering these topics.

  \question[5] \emph{(Shannon)}  Prove that in any perfectly secret shared-key encryption scheme, $|\calK| \ge |\calM|$.

  \begin{solution}
				
  \end{solution}


  \question \emph{(Perfect secrecy.\footnote{Based on a question from Peikert's class\label{fn:peikert}})}  Prove or disprove (giving the
  simplest counterexample you can find) the following statements about
  perfect secrecy for shared-key encryption.  You may use any of the
  facts from class.

  \begin{parts}
    \part[1] There is a perfectly secret encryption scheme for
    which the ciphertext always reveals 99\% of the bits of the key
    $k$ to the adversary.

    \begin{solution}
      
    \end{solution}

    \part[2] There is an encryption scheme that is not perfectly secure, yet
    the adversary cannot guess the key with probability greater than $1/|\calK|$.

    \begin{solution}
      
    \end{solution}

    \part[2] In a perfectly secret encryption scheme, the
    ciphertext is uniformly random.  That is, for every $m \in
    \msgspace$, the probability $\Pr_{k \gets \skcgen}[\skcenc_{k}(m)
    = \bar{c}]$ is the same for every ciphertext $\bar{c} \in
    \ctspace$.

    \begin{solution}
      
    \end{solution}

    \part[5] Perfect secrecy is equivalent to the following
    definition, which says that the adversary cannot determine which
    of two messages was encrypted any better than by random guessing.
    Formally, for any $m_{0}, m_{1} \in \msgspace$, and any function
    $\Adv : \ctspace \to \bit$,
    \[ \Pr_{k \gets \skcgen,\; b \gets \bit}[\Adv(\skcenc_{k}(m_{b}))
    = b] = \frac{1}{2}. \]

    \begin{solution}
      
    \end{solution}

    \part[5] Perfect secrecy is equivalent to the following
    definition, which says that the ciphertext and message are
    independent (as random variables).  Formally, for any probability
    distribution $\calD$ over the message space $\msgspace$ and any
    $\bar{m} \in \msgspace$ and $\bar{c} \in \ctspace$,
    \[ \Pr_{m \gets \calD,\; k \gets \skcgen}[ m = \bar{m} \wedge
    \skcenc_{k}(m) = \bar{c}] = \Pr_{m \gets \calD}[m = \bar{m}] \cdot
    \Pr_{m \gets \calD,\; k \gets \skcgen}[\skcenc_{k}(m) =
    \bar{c}]. \]

    \begin{solution}
      
    \end{solution}
  \end{parts}




  \question \emph{(Encryption schemes with a computationally bounded adversary.)}  
	
	Consider the scenario of an encryption scheme in which Alice wants to send a message to Bob in such a way
	that Eve, who monitors the transmission, cannot read the message.
	
  \begin{parts}
    \part[1] Explain in one sentence why Bob needs to have a secret from Eve.

    \begin{solution}
      
    \end{solution}

    \part[1] Explain in one or two sentences why Alice needs to have a secret from Eve.

    \begin{solution}
      
    \end{solution}

    \part[2] Now assume that Eve is computationally bounded (i.e., is restricted to run in polynomial time in the length
		of the message). Does Bob still need to have a secret from Eve? Does Alice? (your feeling for the latter is enough)

    \begin{solution}
      
    \end{solution}
		
		\moreinfo{18764}

  \end{parts}



  \question \emph{(Working with negligible functions.\cref{fn:peikert})}  Recall that a
  non-negative function $\nu : \N \to \R$ is \emph{negligible} if it
  decreases faster than the inverse of any polynomial (otherwise, we
  say that $\nu$ is \emph{non-negligible}).  More precisely, $\nu(n) =
  o(n^{-c})$ for every fixed constant $c > 0$, or equivalently,
  $\lim_{n \to \infty} \nu(n) \cdot n^{c} = 0$. 

  State whether each of the following functions is negligible or
  non-negligible, and give a brief justification.  In the following,
  $\negl(n)$ denotes some arbitrary negligible function, and
  $\poly(n)$ denotes some arbitrary polynomial in $n$.
	(If you are not comfortable with these notion, read Section 4.2 of Lecture 2
	in Peikert's notes)

  \begin{parts}
    \part[1] $\nu(n) = 1/2^{100 \log n}$.

    \begin{solution}
      
    \end{solution}

    \part[1] $\nu(n) = n^{-\log \log \log n}$.  \hfill {\small
      (Compare with the previous item for ``reasonable'' values of
      $n$.)}

    \begin{solution}
      
    \end{solution}

    \part[1] $\nu(n) = \poly(n) \cdot \negl(n)$. \hfill {\small (State
      whether $\nu$ is \emph{always} negligible, or not necessarily.)}

    \begin{solution}
      
    \end{solution}

    \part[1] $\nu(n) = (\negl(n))^{1/\poly(n)}$. \hfill {\small (Same
      instructions as previous item.)}

    \begin{solution}
      
    \end{solution}

    \part[1] \[ \nu(n) =
    \begin{cases}
      2^{-n} & \text{if $n$ is composite} \\
      100^{-100} & \text{if $n$ is prime.}
    \end{cases}
    \]
    
    \begin{solution}
      
    \end{solution}

  \end{parts}

\renewcommand*{\thefootnote}{$\clubsuit$}
  \question[2] \emph{(Defining one-way functions.\footnote{Questions marked with a club are more open-ended and meant to encourage you to think in preparation for next class. You are not expected to answer correctly. Instead, you are expected to spend time thinking about it.})} Next class we will define the notion of a \emph{one-way function}.
	Informally, this is a function $f:\{0,1\}^* \to \{0,1\}^*$ that is (1) easy to compute, and (2) hard to invert. 
	  \begin{parts}
		\part
		Suggest a way (or ways) to formally define it. 
    \begin{solution}
      
    \end{solution}
  Next, for each of the following functions, say if you think it's one way according to your definition.	\part
	The function that given an $n$-bit string outputs the same string
	with its first half zeroed out.

    \begin{solution}
      
    \end{solution}

	\part The function $f$ on domain $\{1,\ldots,N\} \times \{1,\ldots,N\}$ that maps a pair $(x,y)$ to their product $xy$.

    \begin{solution}
      
    \end{solution}

	\part Choose elements $a_1,\ldots,a_n$ uniformly from $\Z_N$ for $N=2^n$ and define $f:\{0,1\}^n \to \Z_N$ by 
	 \( 
	     f(b_1,\ldots,b_n) = \sum_{i=1}^n b_i a_i.
	 \)

    \begin{solution}
      
    \end{solution}
	 \part Same as previous part, except $a_1,\ldots,a_n$ are chosen uniformly from $\Z_2^n$. 

    \begin{solution}
      
    \end{solution}
	\end{parts}


  \question \emph{(Error-correcting codes (optional, no credit).)} This is a bit off topic, but
	will give you a glimpse to an immensely important topic that also dates back to Shannon's seminal work. These ideas are used in pretty much 
	all digital communication protocols: cell phones, Internet, satellites, etc.
	
  \begin{parts}
    \part Assume we choose $2^{n/20}$ strings from the set $\{0,1\}^n$ uniformly at random. Show that with positive probability (in fact, high probability)
		the Hamming distance (i.e., number of different coordinates) between \emph{any} two strings in the set is more than $n/4$. 
\hint{84542}
		
    \begin{solution}
      
    \end{solution}

    \part Show how Alice can communicate to Bob a message of $k$ bits by sending only $n=20k$ bits in such a way that Bob can recover the message even if an adversary flips up to $n/8$ bits of the communication. Would simply repeating the message 20 times be good enough?
		
    \begin{solution}
      
    \end{solution}

  \end{parts}

\end{questions}

\end{document}

%%% Local Variables: 
%%% mode: latex
%%% TeX-master: t
%%% End: 
